 \documentclass{article}
 \usepackage{amsmath}
 \usepackage{amssymb}
 \usepackage{geometry}
\geometry{legalpaper, portrait, margin=1in}\usepackage{graphicx}
\graphicspath{ {\string~/Documents/PresentationP/} }
\usepackage{hyperref}
 \begin{document} 
 \title{Maths in the Movies: Mathematical Mistakes}
 \author{Ellie Sleightholm and Fred Barker} 
 \maketitle
 \noindent Mathematical mistakes in movies are often overlooked by viewers as their attention is focused solely on the action taking place rather than the mathematical concepts behind it. Despite film companies spending millions of pounds on well-structured storylines, the mathematical details are often disregarded.\\[5pt]
Mathematical inconsistencies within films can be categorised into three separate sections. To begin with, we have the minute inaccuracies that are most definitely overlooked by non-mathematical viewers, these are small errors that make no difference to the films' storyline. In particular, in the episode ‘Court Martial’ of ‘Star Trek’, ‘Captain Kirk’ comments on the auditory sensors of a computer, stating, “we can increase that capability on the order of one to the fourth power” $(1,2)$. The inconsistency here is that “one to the fourth power” is just one, so in effect, there would be no increase in amplification at all.\\[5pt]
In furtherance, the idea of overcomplicating mathematical theories within films can be seen as another inaccuracy. As an example, in the film ‘Good Will Hunting’, we explored the mathematical problem presented on a blackboard that took MIT professors “two years to figure out”. The problem itself is extremely overcomplicated but the maths behind it can be seen at undergraduate level and most university maths students would be able to solve it easily. Here, we see a film purposely make a problem seem complex in order to fascinate the viewers.\\[5pt]
The last categorised mathematical inconsistency can be seen most commonly in action films where they tend to defy all laws of physics and maths in order to create an intense, action-packed scene. For such films to be effective, we, as an audience, want the protagonist to survive. So, the film writers create massive stunts which realistically, would have a minute chance of survival.\\[5pt]
A specific scene we analysed was from ‘Mission Impossible 2’ (4) with ‘Ethan Hunt’ played by Tom Cruise and ‘Sean Ambrose’ played by Dougray Scott. Both characters are on motorbikes facing each other some distance away. They both begin to accelerate at the exact same time towards each other and just before the two bikes collide (after 13 seconds of travelling), both characters jump, travel horizontally, then collide in mid-air.\\[5pt]
The first thing we looked into was the motorcycles' specification. We found that the bike ‘Ethan Hunt’ rides was a Triumph Speed Triple and ‘Sean Ambrose’s bike was a Triumph Daytona 955i (5). The top speeds were 220km/h and 259km/h respectively (6,7). Converting into metres per second we got 61.1 m/s and 71.9 m/s respectively. In addition, we knew that Hunt and Ambrose had weights 80kg and 90kg respectively (8). We then modelled the scene as follows:\\
\includegraphics[width=120mm]{Model}\\[5pt]
We knew both bikes collided after 13 seconds of travelling (4) so our aim was to observe whether both bikes had reached their top speeds (given above) before they collided. In order to solve this, we looked at the bike’s time to accelerate from 0 m/s (0mph) to 26.8m/s (60mph) and then used suvat equations to solve for the time. We found that it took the Triumph Speed Triple 3.3 seconds to reach 60mph (9). We then construct the following suvat equations:
\begin{align*}
&t = 3.3 \hspace{4cm} t=? \\
&u=0 \hspace{4.2cm} u=0\\
&v=26.8 \hspace{3.8cm} v=61.1\\
&a = a \hspace{4.2cm} a=a\\
\end{align*}
Solving for a in the first suvat equation we got:
\begin{align*}
v &= u + at\\
26.8 &=0 + 3.3a\\
a &=8.12 \text{m/s$^2$}
\end{align*}\\[5pt]
Substituting this value of a into the second suvat equation, we got:
\begin{align*}
v &= u + at\\
61.1 &=0 + 8.12t\\
t &=7.52 ~\text{seconds}\\
\end{align*}
Hence, the Triumph Speed Triple and Ethan Hunt reach maximum velocity before colliding. We then constructed similar suvat equations for the Triumph Daytona 955i (which took 3 seconds to reach 60mph) (9) to calculate the time it took to reach top speed:
\vspace{1mm}
\begin{align*}
&t = 3.0 \hspace{4cm} t=? \\
&u=0 \hspace{4.2cm} u=0\\
&v=26.8 \hspace{3.8cm} v=71.9\\
&a = a \hspace{4.2cm} a=a\\
\end{align*}
Solving for a in the first suvat equation we got:
\begin{align*}
v &= u + at\\
26.8 &=0 + 3a\\
a &=8.93 \text{m/s$^2$}\\
\end{align*}
Substituting this value of a into the second suvat equation, we got:
\begin{align*}
v &= u + at\\
71.9 &=0 + 8.93t\\
t &=8.05 ~\text{seconds}\\
\end{align*} Therefore, we see that both bikes and characters reach top speed before the collision. With the information calculated and taking direction to the right as positive, we can remodel as follows:\\
\includegraphics[width=110mm]{Model2}\\[5pt]
We had collected all the information necessary, so were able to calculate the force on the characters using Impulse. Impulse is given by the equation below (10) :\\[3pt]

Impulse = $Force\times Time Interval =Mass x (Change in Velocity)$\\[10pt]
Given that the time interval was 0.015 seconds and the masses are defined above, we were able to calculate the change in velocity. Since both characters coalesce, we needed to calculate their collective final speed. We did so as follows:\\[5pt]
Initial Momentum = $m\textsubscript{1}u\textsubscript{1} + m\textsubscript{2}u\textsubscript{2}$ = Final Momentum = (m\textsubscript{1} + m\textsubscript{2})v\textsubscript{1}\\[5pt]
Given they coalesce:
 \begin{align*}
m\textsubscript{1}u\textsubscript{1} + m\textsubscript{2}u\textsubscript{2} &= (m\textsubscript{1} + m\textsubscript{2})v\textsubscript{1}\\
(80 ~\text{x} ~61.1) + (90~ \text{x} -71.9) &= (80 + 90)v\\
-972 &= 170v\\
v &= 5.72 \text{m/s} ~\text{(in the negative/left direction)}\\
\end{align*}
We were then able to calculate the Force with the equation above:\\

Force = $\dfrac{Mass x (Change in Velocity)}{Time Interval}$\\

Force of Ethan Hunt on Sean Ambrose = $ \dfrac{80~ \text{x} ~(61.1 + 5.72)}{0.015}$ = $356,373$N\\

Force of Sean Ambrose on Ethan Hunt = $ \dfrac{90~ \text{x} ~(71.9 - 5.72)}{0.015}$ = $397,080$N\\[6pt]
Furthermore, given that the entire impact was on their upper bodies, we calculated the pressure on their upper bodies at this time. Given that Ethan Hunts' upper body area was 0.34 m$^2$ and Sean Ambroses' upper body area was 0.35m$^2$, we used the following equation (11) :\\[6pt]
Pressure = $\dfrac{\text{Force}}{\text{Area}}$\\[5pt]
Hence, the pressure on Ethan Hunt by Sean Ambrose = $\dfrac{397,080}{0.34} = 1,167,882$ N/m$^2$\\[5pt]
Pressure on Sean Ambrose by Ethan Hunt = $\dfrac{356,373}{0.35} = 1,018,208$ N/m$^2$\\ [5pt]
In car crash studies a pressure of 350,000 N/m$^2$ on a human body resulted in a 50-50 chance survival rate. Those who survived such pressures had massive internal trauma. Since the pressure on our two characters excessively surpasses that of 350,000, we can conclude that both Ethan Hunt and Sean Ambrose would have suffered extreme trauma after the collision and would have most likely died. Therefore, the chance of the two characters getting up to continue fighting after they hit the ground is unrealistic. Overall, the likelihood of surviving this particular stunt would have been impossible and the scene is therefore riddled with massive mathematical mistakes throughout. 
\vspace{5mm}\\
 \textbf{References}:\\
 
 \noindent (1) RF Cafe. [Online]. 1996. [Date Accessed: 3rd December 2018]. Available from: \url{http://www.rfcafe.com/miscellany/humor/1-to-4th-power-amplification-star-trek-court-martial.htm}\\
 (2) Mathologer. 10 of the Greatest Math Movie Bloopers. [Online]. 2015. [Date Accessed: 3rd December]. Available from: \url{https://www.youtube.com/watch?v=zBuykQHFQ1Q}\\
 (3) Numberphile: The Problem in Good Will Hunting - Numberphile. [Online] 2013. [Date Accessed: 3rd December]. \url{https://www.youtube.com/watch?v=iW_LkYiuTKE&t=206s}
 (4) OnlyMovies200. Mission Impossible II - Motorcycle chase. [Online] 2016. [Date Accessed: 2nd December]. Available from: \url{HDhttps://www.youtube.com/watch?v=K2oKEqtQFyc}\\
 (5) BikeBound. [Online]. [Date Accessed: 2nd December] Available from: \url{http://www.bikebound.com/2018/07/14/what-are-the-motorcycles-in-mission-impossible-2/}\\
 (6) Wikipedia. [Online]. 2017. [Date Accessed: 2nd December]. Available from: \url{https://en.wikipedia.org/wiki/Triumph_Daytona_955i}\\
 (7) Quora. [Online]. 2018. [Date Accessed 2nd December]. Available from: \url{https://www.quora.com/What-is-the-top-speed-of-a-Triumph-Street-Triple}\\
 (8) Popular Science. [Online]. 2007. [Date Accessed: 2nd December]. Available from: \url{https://www.popsci.com/entertainment-gaming/gallery/2007-09/hollywood-physics}\\
 (9) Zero to 60 Times. [Online] [Date Accessed: 2nd December]. Available from: \url{https://www.zeroto60times.com/vehicle-make/triumph-motorcycles-0-60-mph-times/}\\
 (10) ZonaLand Education. [Online]. [Date Accessed: 3rd December 2018]. Available from: \url{http://zonalandeducation.com/mstm/physics/mechanics/momentum/introductoryProblems/momentumSummary2.html}\\
 (11) Wikipedia. [Online]. 2017. [Date Accessed: 2nd December]. Available from: \url{https://en.wikipedia.org/wiki/Pressure}
 
\end{document}
