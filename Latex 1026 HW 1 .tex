\documentclass{article}
\usepackage{amsmath}
\usepackage{amssymb}
\usepackage{geometry}
\geometry{legalpaper, portrait, margin=0.75in}


\begin{document}
\large{\textbf{Question 4 - Homework 1 - Sets, Series and Sequences}
\vspace{5 mm}

Let A $= \{ \dfrac{n}{m} : n$ $\in$ $\mathbb{N}$, $m$ $\in$ $\mathbb{N}$, $m > n \}.$

\vspace{5mm}

(a) Prove that inf $A = 0$
\vspace{5mm}

\textbf{Solution:}

\vspace{5mm}

To show that $0$ is a infimum of $A$, we must prove the following:

\vspace{2mm}

(i) That $0$ is a lower bound in $A$ such that $\forall x \in A, x \geqslant 0$.

\vspace{2mm}

(ii) That for all $b$, with $b > 0$, b is not a lower bound such that $\forall b, b > 0, \exists x \in A, b > x$.

\vspace{4mm}

So, for (i):

\vspace{2mm}

Take $n = 1$ and $m \rightarrow \infty$ since $m > n$, then, $\dfrac{n}{m} \rightarrow 0$ which is the lowest possible value for 

$A$.

\vspace{2mm}

Taking the maximum value of as $n \rightarrow \infty$, and $m = n+1$, $\dfrac{m}{n} \rightarrow 1$.

\vspace{2mm}

So, A takes the values $(0, 1)$. Hence, 0 is always less than these values so condition (i) is 

\vspace{2mm}

satisfied and 0 is a lower bound. 

\vspace{5mm}

For condition (ii):

\vspace{2mm}

Let $\epsilon > 0$ be given. By the Archimedean Property, we can find an $n$ $\in$ $\mathbb{N}$, such that 

\vspace{2mm}

$\dfrac{1}{n} < \epsilon$. As $m > n$, for all $n > 0$, we find that:

 \begin{align*}
\dfrac{1}{m} < \dfrac{1}{n} < \epsilon \implies \dfrac{1}{m} < \epsilon
\end{align*}

\vspace{2mm}

We know that when $n = 1$, $\dfrac{n}{m} = \dfrac{1}{m}$, so $\dfrac{1}{m} \in A$. So,
\begin{align*}
0 + \epsilon < x,\hspace{2mm} \text{for} \hspace{2mm} x \in A
\end{align*}

\vspace{2mm}

Thus, inf $A = 0$ as both condition (i) and (ii) are satisfied. 

\vspace{6mm} 

(b) Sup $A = 1$
\vspace{5mm}

\textbf{Solution:}

\vspace{5mm}

To show that $1$ is a supremum of $A$, we must prove the following:

\vspace{2mm}

(i) That $1$ is an upper bound in $A$ such that $\forall x \in A, x \leqslant 1$.

\vspace{2mm}

(ii) That for all $b$, with $b < 1$, b is not an upper bound such that $\forall b, b < 1, \exists x \in A, b < x$.

\vspace{2mm}

So, for (i):

\vspace{2mm}

Take $n = 1$ and $m \rightarrow \infty$ since $m > n$, then, $\dfrac{n}{m} \rightarrow 0$ which is the lowest possible value for 

$A$.

\vspace{2mm}

Taking the maximum value of as $n \rightarrow \infty$, and $m = n+1$, $\dfrac{m}{n} \rightarrow 1$

\vspace{2mm}

So, we can conclude that A takes the values $(0, 1)$. Hence, 1 is always greater than these 

\vspace{2mm}

values, so condition (i) is satisfied and 1 is an upper bound. 

\vspace{5mm}

\newpage

For condition (ii):

\vspace{2mm}

Let $\epsilon >0$ be given. By the Archimedean Property, we can find an $n$ $\in$ $\mathbb{N}$, such that 

\vspace{2mm}

$\dfrac{1}{n} < \epsilon$. As $m > n$, for all $n > 0$, we find that:
\begin{align*}
\dfrac{1}{m} < \dfrac{1}{n} < \epsilon \implies - \epsilon < - \dfrac{1}{m}
\end{align*}

\vspace{2mm}

Adding $1$ to both sides, we get:
\begin{align*}
1 - \epsilon < 1 - \dfrac{1}{m}
\end{align*}

\vspace{2mm}

Then, when $m = n + 1$ because $m > n$, we get $\dfrac{n}{m} = 1 - \dfrac{1}{m}$, so $1 - \dfrac{1}{m} \in A$. Then, 

\begin{align*}
1 - \epsilon < x \hspace{2mm} \text{for} \hspace{2mm} x \in A
\end{align*}

\vspace{2mm}

Thus, sup $A = 1$ as both condition (i) and (ii) are satisfied. 

















\end{document}